%Template credit: Jan-Willem Steeb, NRAO

\documentclass[12pt,a4paper]{article}

\usepackage{graphics,graphicx}
\usepackage[%
    font={small,sf},
    labelfont=bf,
    format=hang,    
    format=plain,
    margin=0pt,
    width=0.8\textwidth,
]{caption}
\usepackage[list=true]{subcaption}
\usepackage{amsmath}
\usepackage{amssymb}
\usepackage{bm}
\usepackage{listings}
\usepackage{hyperref}
\usepackage{lmodern}  
\usepackage{amsmath}  
\usepackage{xcolor}   
\lstset{
  basicstyle=\ttfamily,
  columns=fullflexible,
  frame=single,
  breaklines=true,
  postbreak=\mbox{\textcolor{red}{$\hookrightarrow$}\space},
}

\textheight=247mm
\textwidth=180mm
\topmargin=-7mm
\oddsidemargin=-10mm
\evensidemargin=-10mm
\parindent 10pt

%%%%%%%%%%%%%%%%%%%%%
%%%%% Custom Commands %%%%
%%%%%%%%%%%%%%%%%%%%%
\newcommand{\vb}[1]{\text{\textbf{#1}}} %make non special characters bold in math mode, used for vectors and matrices
\newcommand{\n}[1]{\text{{#1}}} %removes math styling, useful for subscripts

%Mathematic Functions
\DeclareMathOperator*{\argmax}{arg\,max}
\DeclareMathOperator*{\argmin}{arg\,min}
\DeclareMathOperator*{\mmid}{mid}
\DeclareMathOperator*{\at}{arctan2}

%%%%%%%%%%%%%%%%%%%%%
%%%%% Start of document %%%%% 
%%%%%%%%%%%%%%%%%%%%%

\begin{document}
\pagestyle{plain}
\pagenumbering{arabic}
 
%%%%%%%%%%%%%
%%%%% Title  %%%%%
%%%%%%%%%%%%%%

\begin{center}
{\Large{\bf{  Data Comb 2019: QAC Memo \\  }}} 

\end{center}
\bigskip

\centerline{Peter Teuben (University of Maryland)}

\centerline{\today}
\bigskip

%%%%%%%%%%%%%%%%%%%%%%%%%%%
%% Please include this line
\noindent \textit{This memo was prepared as part of the workshop ``Improving Image Fidelity on Astronomical Data: Radio Interferometer and Single-Dish Data Combination,'' held on 12-16 Aug 2019 at the Lorentz Center in Leiden, The Netherlands.}
%%%%%%%%%%%%%%%%%%%%%%%%%%%

\section{Introduction}

The Quick Array Combination (QAC) toolkit came first into existence to shortcut some tedious organizational python code that seemed to be always needed
writing codes to run simulations, combinations and comparisons. QAC follows the strategy used in the CASA simulator, in which
datasets are maintained within a project directory with predictable names.
In addition, QAC would fix and containerize some CASA issues we ran into during the TP2VIS development. Thus QAC is a python layer to CASA,
and QAC and CASA commands can be used interchangeably.

QAC is also described in ngVLA memo 59.

\section{Installing QAC}

Typically QAC is installed by adding the line
\begin{lstlisting}[language=bash]
    execfile(os.environ['HOME'] + '/.casa/QAC/casa.init.py')
\end{lstlisting}

to your \verb+~/.casa/init.py+ file. You can either place a symbolic link to the location where you keep QAC, or place it directly
inside the \verb+~/.casa+ directory.  QAC has optional methods to install other useful CASA components, such
as the public tp2vis release, SDINT, SD2VIS etc. More information can be found in the QAC distribution.

\section{Using QAC}

Given a CASA script \verb+foo.py+, you can execute this in two ways
First, directly from the Unix shell (e.g. bash), and optionally
pass it command line arguments if the script knows how to do it:
 
\begin{lstlisting}[language=bash]
    % casa --nogui -c foo.py pixel=0.1 > foo.log 2>&1
\end{lstlisting}

This redirects the screen output to \verb+foo.log+.  Secondly, you can execute it directly from within CASA's ``ipython'' session

\begin{lstlisting}[language=Python]
CASA <1>:   execfile("foo.py")
\end{lstlisting}

but this has the disadvantage that no command line parameters can be given. All your parameters need to be hardcoded
in \verb+foo.py+ \footnote{Python's execfile only works in Python 2, and will thus disappear from CASA once it changes to Python 3}

% In the RICA package you can find a 3rd variant using the ipython {\tt run} command - but this is virgin python that needs casa imports.


\subsection{A poor man's command line parser}

You can also use Python's {\tt argparse} or use QAC's poor man's (very limited error checking).  You first define
the keywords with a default, then parse the command line, after which their values have been overriden from the ones on
the command line.

\begin{lstlisting}[language=Python]
  # set defaults for parameters
  pdir    = 'sky1'
  imsize  = 2048
  pixel   = 0.05
  niter   = [0,1000,4000]

  # parse the ``keyword=value'' based command line parameters
  import sys
  for arg in qac_argv(sys.argv):   
      exec(arg)

  # show the new values of the parameters
  print(pdir,imsize,pixel,niter)

\end{lstlisting}

after which you can use it as in the following example (note the double quoting to escape the shell meta characters):

\begin{lstlisting}[language=bash]
  % casa --nogui -c foo.py pdir='"test1"' imsize=1024 pixel=0.1 niter='[0,1000]'  >foo.log 2>&1
\end{lstlisting}

\subsection{Using ``make''}

The classic ``make'' command in Unix uses a \verb+Makefile+ where the recipes (normally called targets) are stored. This has the advantage that
it self-documents all the work needed or was done in a particular directory. Running CASA scripts via a Unix commandline also has the
advantage that log files can be compared, and a  graphical difference between two logfiles.
Here is the Makefile entry for running the benchmark ``bench'':

\begin{lstlisting}[language=make]
  # pick your CASA or use a smart $PATH
CASA = time /Applications/CASA.app/Contents/MacOS/casa --nogui -c
CASA = time /opt/casa-release-5.6.0-60.el7/bin/casa --nogui -c
CASA = time casa --nogui -c
  
bench:
    $(CASA) bench.py > bench.log 2>&1
    cp bench.log bench
\end{lstlisting}

and here an example\footnote{``astroload'' is a personal alias to load in different versions of CASA, YMMV} of comparing the
benchmark for two difference version of CASA:

\begin{lstlisting}[language=bash]
  astroload -v 5.6 casa
  make bench
  mv bench bench_56
  astroload -v 5.5 casa
  make bench
  mv bench bench_55
  tkdiff bench_56/bench.log bench_55/bench.log
  casa -c 'qac_math("diff1.im","bench_55/clean/tpint.image","-","bench_56/clean/tpint.image")'
  casaviewer diff1.im
\end{lstlisting}

In my experience looking at the difference between logfiles and maps this way, a lot of obvious problems can be uncovered more quickly.

\section{Example Script}

Here is a simple example of combining Visibilities and TP data, the QAC style:

\begin{lstlisting}[language=Python]
  # Datasets we need (assumed to be all gridded on the same spectral grid)
  ms12 = '12m.ms'   # 12m visibilities
  ms7  = '7m.ms'    # 7m visibilities
  tpim = 'tp.im'    # TP image cube

  
  # Do all the work in the directory "M100" (it will be created, if exists, all contents removed)
  qac_test('M100')
  
  # get the pointings from the 12m data
  qac_ms_ptg(ms12,'M100/12m.ptg')

  # generate the visibilities from the TP cube, using a known rms of 0.15 Jy/beam in the TP cube
  qac_tp_vis('M100',tpim,'M100.ptg',rms=0.15)

  # tclean the 7m, 12m and TP visibilities together
  qac_clean('M100/clean1','M100/tp.ms',[ms12,ms7],800,0.5,phasecenter='J2000 12h22m54.900s +15d49m15.000s',niter=[0,10000])

  # feather the 7m+12m with the TP image  [@todo the regrid]
  qac_feather('M100','M100/clean1/tpint_2.image.pbcor','tp.im',regrid=True)

  # SSC
  qac_ssc()

  # SDVIS
  qac_sd_int('M100',...)

\end{lstlisting}


\section{Combination Methods}

\subsection{Method 1: Feather (combining after deconvolution)}

There are currently two ``feather'' type implementations in QAC:

\subsubsection{CASA feather}

\begin{lstlisting}[language=Python]
def qac_feather(project, highres=None, lowres=None, label="", niteridx=0, name="dirtymap"):
    """
    Feather combination of a highres and lowres image. See also qac_ssc()

    project  --  typical  "sky3/clean2", somewhere where tclean has run
    highres  --  override default, needs full name w/ its project
    lowres   --  override default, needs full name w/ its project
    
    If the standard workflow is used, project contains the correctly named
    dirtymap.image and otf.image from qac_clean1() and qac_tp_otf() resp.
    @todo figure out if a manual mode will work

    Typical use in a simulation:
\end{lstlisting}

CASA's feather() implemention can also be used, if you know all your filenames, there is no need to use the QAC version
in that case.

\subsubsection{Faridani's SSC}

This method is also available in NOD3 as ``immerge''. ( check ). The QAC version assumes a directory structure.

\begin{lstlisting}[language=Python]
def qac_ssc(project, highres=None, lowres=None, f=1.0, sdTel = None, regrid=True, cleanup=True, label="", niteridx=0, name="dirtymap"):
    """
        project --    directory in which all work will be performed
        highres --    high res (interferometer) image
        lowres  --    low res (SD/TP) image
        sdTEL   --    if not provided, sdFITS must contain the telescope
        regrid  --    if you are sure of the same WCS, set this to False
    """
\end{lstlisting}

\subsection{Method 2: Joint Deconvolution (combining before deconvolution)}

Again, we have several implementations available:

\subsubsection{tp2vis}

\begin{lstlisting}[language=Python]
def qac_tp_vis(project, imagename, ptg=None, pixel=None, phasecenter=None, rms=None, maxuv=10.0, nvgrp=4, fix=1, deconv=True, winpix=0, **line):    
           
    """
      project:       identifying (one level deep directory) name within which all files are places
      imagename:     casa image in RA-DEC-POL-FREQ order (fits file is ok too)
      ptg            filename of list of pointings
      pixel:         pixel size, in arcsec, if to be overriden from the input map. 
      phasecenter    Defaults to mapcenter
      rms            if set, this is the TP cube noise to be used to set the weights
      maxuv          maximum uv distance of TP vis distribution (in m)  [10m] 
      nvgrp          Number of visibility group (nvis = 1035*nvgrp)
      fix            Various fixes such that tclean() can handle a list of ms [fix=1]
      deconv         Use the deconvolved map as model for the simulator.
      winpix         Tukey window [0]
      line           Dictionary of tclean() parameters, usually the line parameters are useful, e.g.
                     line = {"restfreq":"115.271202GHz","start":"1500km/s", "width":"5km/s","nchan":5}
    """
\end{lstlisting}

\subsubsection{sdvis}

From the Nordic ARC. Handles one pointing only. Not tested in this workshop.

\begin{lstlisting}[language=Python]

  def qac_sd_vis(**kwargs):
  """
  SD2vis from the Nordic ARC Tools
  """

\end{lstlisting}


\subsubsection{sdint}

Currently under evaluation in the CASA developers team.  Being implemented in QAC as well.
Minor issue: does not handle spectral axis being different in MS and IM yet, so some extra pre-processing
is needed.

\begin{lstlisting}[language=Python]

def qac_sd_int(sdimage, vis, sdpsf, **kwargs):
    """
    sdimage:
    vis:
    sdpsf:
    """

\end{lstlisting}


\subsection{Method 3 (hybrid startmodel) (combining during deconvolution)}

TBD. Partially tested. Needs a proper QAC implementation the way Kauffmann described it, but our qac\_clean() supports the
startmodel= keyword.

\bigskip
{\bf Acknowledgements:} partially written under the influence of good Dutch coffee and apple pie in ``Brasserie Buitenhuis''.

\clearpage
\section{Appendix: Unresolved Issues}

Although you may find these also in the github issue tracker for dc2019, we list here some issues to be resolved related to QAC/TP2VIS
and possible need fixes in CASA as well:

\begin{enumerate}

\item imtrans() on the TP does not preserve flux in benchmark (1 on 474)
\item differencing two runs on the benchmark shows some numbers that vary, where our seed=123 should have fixed everything?
  The flux is conserved though. Something clearly odd in tp2vis.
\item despite gridding the benchmark data from 1400 to 1745 km/s in steps of 5 km/s, we get unpredictable bad first or last channel
  (either 0 or combinations fail in those channels). Is this a CASA roundoff problem?  Data are all supposed to be in LSRK, so no more
  Doppler shifting should be needed.

\end{enumerate}

\section{tips}

\begin{enumerate}

\item authors of scripts for community use should list how much memory and diskspace it will take, and some idea of the CPU time

\item it helps to develop a script that with a few obvious changes to the parameters at the top will run quickly, vs. the full run.
  Examples could be to easily set fewer channels, fewers pointings etc.

\item related to the previous item,
  if you develop a script inside of CASA, the command \verb+%history -f mysession.py+ can be used to save your session and
  transform into a re-usable script instead of cutting and pasting commands over and over again.
  
\end{enumerate}


\section{new features}

Many small changes to QAC and related software, but worth noting
from this week are:

\begin{enumerate}

\item tp2vis should check if outfile exists, and not overwrite it

\item provisionary sdint interface added 
  
\end{enumerate}

\end{document}
