%Template credit: Jan-Willem Steeb, NRAO

\documentclass[12pt,a4paper]{article}

\usepackage{graphics,graphicx}
\usepackage[%
    font={small,sf},
    labelfont=bf,
    format=hang,    
    format=plain,
    margin=0pt,
    width=0.8\textwidth,
]{caption}
\usepackage[list=true]{subcaption}
\usepackage{amsmath}
\usepackage{amssymb}
\usepackage{bm}
\usepackage{listings}
\usepackage{hyperref}
\usepackage{lmodern}  
\usepackage{amsmath}  
\usepackage{xcolor}   
\lstset{
  basicstyle=\ttfamily,
  columns=fullflexible,
  frame=single,
  breaklines=true,
  postbreak=\mbox{\textcolor{red}{$\hookrightarrow$}\space},
}

\textheight=247mm
\textwidth=180mm
\topmargin=-7mm
\oddsidemargin=-10mm
\evensidemargin=-10mm
\parindent 10pt

%%%%%%%%%%%%%%%%%%%%%
%%%%% Custom Commands %%%%
%%%%%%%%%%%%%%%%%%%%%
\newcommand{\vb}[1]{\text{\textbf{#1}}} %make non special characters bold in math mode, used for vectors and matrices
\newcommand{\n}[1]{\text{{#1}}} %removes math styling, useful for subscripts

%Mathematic Functions
\DeclareMathOperator*{\argmax}{arg\,max}
\DeclareMathOperator*{\argmin}{arg\,min}
\DeclareMathOperator*{\mmid}{mid}
\DeclareMathOperator*{\at}{arctan2}

%%%%%%%%%%%%%%%%%%%%%
%%%%% Start of document %%%%% 
%%%%%%%%%%%%%%%%%%%%%

\begin{document}
\pagestyle{plain}
\pagenumbering{arabic}
 
%%%%%%%%%%%%%
%%%%% Title  %%%%%
%%%%%%%%%%%%%%

\begin{center}
{\Large{\bf{  Data Comb 2019: QAC Memo \\  }}} 

\end{center}
\bigskip

\centerline{Peter Teuben (University of Maryland)}

\centerline{\today}
\bigskip

%%%%%%%%%%%%%%%%%%%%%%%%%%%
%% Please include this line
\noindent \textit{This memo was prepared as part of the workshop ``Improving Image Fidelity on Astronomical Data: Radio Interferometer and Single-Dish Data Combination,'' held on 12-16 Aug 2019 at the Lorentz Center in Leiden, The Netherlands.}
%%%%%%%%%%%%%%%%%%%%%%%%%%%

\section{Introduction}

The Quick Array Combination (QAC) toolkit came first into existence to shortcut some tedious organizational python code that seemed to be always needed
writing codes to run combinations and comparisons, in particular in simulations. It follows the strategy used in the CASA simulator, in which
datasets are maintained within a project directory. In addition, QAC would fix and containerize some CASA issues. Thus QAC is a python layer for CASA,
and QAC and CASA commands can be used interchangeably.

QAC is also described in ngVLA memo 59.

\section{Installing QAC}

Do we need this?  Or just refer to the github repo?

\section{Using QAC}

With a CASA script \verb+foo.py+, you can execute this directly from the Unix shell (e.g. bash) 

\begin{lstlisting}[language=bash]
    % casa --nogui -c foo.py pixel=0.1 > test1.log 2>&1
\end{lstlisting}

which preserves the screen output in \verb+test.log+.  Or you can execute it directly from within CASA's ``ipython'' session

\begin{lstlisting}[language=Python]
CASA <1>:   execfile("foo.py")
\end{lstlisting}

but this has the disadvantage that no command line parameters can be given. All your parameters need to be hardcoded
in \verb+foo.py+ \footnote{Python's execfile command only works in Python 2}.

\subsection{A poor man's command line parser}

You can also use Python's {\tt argparse}   (ref....) or use QAC's poor man's (very limited error checking).  You first define
the keywords with a default, then parse the command line, after which their values have been overriden from the ones on
the command line.

\begin{lstlisting}[language=Python]
  # set defaults
  pdir    = 'sky1'
  imsize  = 2048
  pixel   = 0.05
  niter   = [0,1000,4000]

  # parse the command line
  import sys
  for arg in qac_argv(sys.argv):   
      exec(arg)

  # use the new values
  print(pdir,imsize,pixel,niter)

\end{lstlisting}

after which you can use it as in the following example (note the double quoting to escape the shell meta characters):

\begin{lstlisting}[language=bash]
  % casa --nogui -c foo.py pdir='"test1"' imsize=1024 pixel=0.1 niter='[0,1000]'  > test1.log 2>&1
\end{lstlisting}

\subsection{Using make}

The classic ``make'' command in Unix uses a Makefile where the recipes (normally called targets) are stored. This has the advantage that

\begin{lstlisting}[language=bash]
  astroload -v 5.6 casa
  make bench
  mv bench bench_56
  astroload -v 5.5 casa
  make bench
  mv bench bench_55
  tkdiff bench_56/bench.log bench_55/bench.log
  casa -c 'qac_math("diff1.im","bench_55/clean/tpint.image","-","bench_56/clean/tpint.image")'
  casaviewer diff1.im
\end{lstlisting}


\section{Combination Methods}

\subsection{Method 1: Feather}

There are currently two ``feather'' type implementations in QAC:

\subsubsection{CASA feather}

\begin{lstlisting}[language=Python]
def qac_feather(project, highres=None, lowres=None, label="", niteridx=0, name="dirtymap"):
    """
    Feather combination of a highres and lowres image. See also qac_ssc()

    project  --  typical  "sky3/clean2", somewhere where tclean has run
    highres  --  override default, needs full name w/ its project
    lowres   --  override default, needs full name w/ its project
    
    If the standard workflow is used, project contains the correctly named
    dirtymap.image and otf.image from qac_clean1() and qac_tp_otf() resp.
    @todo figure out if a manual mode will work

    Typical use in a simulation:
\end{lstlisting}

\subsubsection{Faridani's SSC}

\begin{lstlisting}[language=Python]
def qac_ssc(project, highres=None, lowres=None, f=1.0, sdTel = None, regrid=True, cleanup=True, label="", niteridx=0, name="dirtymap"):
    """
        project --    directory in which all work will be performed
        highres --    high res (interferometer) image
        lowres  --    low res (SD/TP) image
        sdTEL   --    if not provided, sdFITS must contain the telescope
        regrid  --    if you are sure of the same WCS, set this to False
    """
\end{lstlisting}

\subsection{Method 2: Joint Deconvolution}

Again, we have several implementations available:

\subsubsection{tp2vis}

\begin{lstlisting}[language=Python]
def qac_tp_vis(project, imagename, ptg=None, pixel=None, phasecenter=None, rms=None, maxuv=10.0, nvgrp=4, fix=1, deconv=True, winpix=0, **line):    
           
    """
      Simple frontend to call tp2vis() 
    
    
      _required_keywords:
      ===================
      project:       identifying (one level deep directory) name within which all files are places
      imagename:     casa image in RA-DEC-POL-FREQ order (fits file is ok too)
      ptg            1) Filename with pointings (ptg format) to be used
                     2_ List of (string) pointings
                     If none specified, it will currently return, but there may be a
                     plan to allow auto-filling the (valid) map with pointings.
                     A list of J2000/RA/DEC strings can also be given.
    
    
      _optional_keywords:
      ===================
    
      pixel:         pixel size, in arcsec, if to be overriden from the input map. Default: None
                     Note we won't allow you to change the imsize.
      phasecenter    Defaults to mapcenter (note special format)
                     e.g. 'J2000 00h48m15.849s -73d05m0.158s'
      rms            if set, this is the TP cube noise to be used to set the weights
      maxuv          maximum uv distance of TP vis distribution (in m)  [10m] 
      nvgrp          Number of visibility group (nvis = 1035*nvgrp)
      fix            Various fixes such that tclean() can handle a list of ms.
                     ** this parameter will disappear or should have default 1
                     0   no fix, you need to run mstransform()/concat() on the tp.ms
                     1   output only the CORRECTED_DATA column, remove other *DATA*, POINTING table also removed
                     2   debug mode, keep all intermediate MS files
                     @todo   there is a flux difference between fix=0 and fix=1 in dirtymap
      deconv         Use the deconvolved map as model for the simulator
                     Within CASA you can use use deconvolve() to construct a Jy/pixel map.

      winpix         Tukey window [0]

      line           Dictionary of tclean() parameters, usually the line parameters are useful, e.g.
                     line = {"restfreq":"115.271202GHz","start":"1500km/s", "width":"5km/s","nchan":5}
    """
\end{lstlisting}

\subsubsection{sdvis}

From the Nordic ARC. Handles one pointing. Not tested in this workshop.

\begin{lstlisting}[language=Python]

  def qac_sd_vis(**kwargs):
  """
  SD2vis from the Nordoc Tools
  """

\end{lstlisting}


\subsubsection{sdint}

Currently under evaluation in the CASA developers team.  Being implemented in QAC as well.
Minor issue: did not handle spectral axis being different in MS and IM.

\begin{lstlisting}[language=Python]

def qac_sd_int(sdimage, vis, sdpsf, **kwargs):
    """
    QAC interface to sdint
    """


\end{lstlisting}


\subsection{Method 3 (hybrid startmodel)}

TBD. Partially tested. Needs a proper QAC implementation the way Kauffmann described it.

\bigskip
{\bf Acknowledgements:} partially written under the influence of good Dutch coffee and apple pie in ``Brasserie Buitenhuis''.

\clearpage
\section{Appendix: ???}

\begin{lstlisting}[language=Python]

\end{lstlisting}



\end{document}
